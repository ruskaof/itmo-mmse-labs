\documentclass[12pt]{article}

\usepackage{xcolor}
\usepackage[unicode, pdftex]{hyperref}

\usepackage{float}
\usepackage{listings}
\usepackage{mathtools}
\usepackage[russian]{babel}
\usepackage{csquotes}

\usepackage{enumitem} % for [itemsep=0.5ex]
\everymath{\displaystyle}

\usepackage{colortbl}
\usepackage{graphicx}
\graphicspath{ {./} }
\usepackage{geometry}
\geometry{
  a4paper,
  top=25mm, 
  right=30mm, 
  bottom=25mm, 
  left=30mm
}

\begin{document}
\pagenumbering{gobble}
\begin{center}
    Университет ИТМО
\end{center}
\vspace{0,5cm}

\begin{center}
Факультет программной инженерии и компьютерной техники
\end{center}

\vspace{1cm}


\begin{center}
    
    \textbf{Лабораторная работа № 1}\\
    по дисциплине «Методы и средства программной инженерии» \\

\end{center}

\vspace{10cm}

\begin{flushright}
    Выполнил:\\
    Студент группы P32302 \\
    Русинов Дмитрий Станиславович\\
    Преподаватель: \\
    Барсуков илья Александрович\\
\end{flushright}

\vspace{3cm}

\begin{center}
    Санкт-Петербург\\
    2023
\end{center}

\newpage
\pagenumbering{arabic} % start page numbering again

\tableofcontents

\newpage

\section*{Задание}
\addcontentsline{toc}{section}{Задание}
 Вариант №149: \enquote{Big Puzzle} - пазлы онлайн. Коллекция пазлов онлайн. Подбор пазла по тематике и сложности. Раздел популярных пазлов. Рейтинг игроков - \url{http://bigpuzzle.ru}

 Составить список требований, предъявляемых к разрабатываемому веб-сайту (в соответствии с вариантом). Требования должны делиться на следующие категории:
 \begin{itemize}[itemsep=0.5ex]
     \item Функциональные.
        \begin{itemize}[itemsep=0.5ex]
            \item Требования пользователей сайта.
            \item Требования владельцев сайта.
        \end{itemize}
    \item Нефункциональные.
 \end{itemize}

Требования необходимо оформить в соответствии с шаблонами RUP (документ SRS - Software Requirements Specification). Для каждого из требований нужно указать его атрибуты (в соответствии с методологией RUP), а также оценить и аргументировать приблизительное количество часов, требующихся на реализацию этого требования.

Для функциональных требований нужно составить UML UseCase-диаграммы, описывающие реализующие их прецеденты использования.



\subsection*{Отчёт по лабораторной работе должен содержать:}
\begin{enumerate}[itemsep=0.5ex]
    \item Документ Software Requirements Specification, содержащий список требований к сайту.
    \item UseCase-диаграммы прецедентов использования, реализующих функциональные требования.
    \item Выводы по работе.
\end{enumerate}

\subsection*{Вопросы к защите лабораторной работы:}
\begin{enumerate}[itemsep=0.5ex]
    \item Методологии разработки ПО. Унифицированный процесс.
    \item Требования и их категоризация. Атрибуты требований.
    \item Язык UML.
    \item Прецеденты использования. UseCase-диаграммы - состав, виды связей.
\end{enumerate}

\section{Introduction}
\subsection{Purpose}
Данный документ формирует спецификацию требований к продукту BigPuzzle.ru, формирует функциональные и нефункциональные требования к нему.
\subsection{Scope}
Цель данного документа состоит в том, чтобы описать требования к веб-сайту BigPuzzle.ru. Он будет использоваться сотрудниками для создания продукта, который полностью соответствует описанным в документе требованиям
\subsection{Definitions, Acronyms and Abbreviations}
\begin{itemize}
    \item ПК, пк - персональный компьютер
    \item Соц. сети - социальные сети
    \item email - электронная почта - сервис для отправки и приёма электронных писем
    \item тег - именованная метка
    \item пагинация - паттерн отображения большого количества списочных или табличных данных посредством загрузки только их части единовременно
    \item пиастры - название внутриигровой валюты (рейтинга)
    \item vip-статус - статус пользовательского аккаунта, позволяющий пользоваться сайтом без рекламы
    \item Uptime - длительность или доля времени работы системы без сбоев
    \item Backup - копия данных системы, хранящаяся независимо от основных данных, позволяющая восстановить основные данные в экстренных случаях
    \item js - язык программирования JavaScript
    \item HTML - язык гипертекстовой разметки HyperText Markup Language
    \item CSS - язык описания стилей страницы Cascading Style Sheets
    \item HTTP - протокол передачи данных HyperText Transfer Protocol
\end{itemize}
\subsection{References}
BigPuzzle.ru - основной сайт системы
\subsection{Overview}
\begin{itemize}
    \item раздел 2 - общее описание факторов системы
    \item раздел 3 - описание всех требований к разрабатываемой системе
    \item раздел 4 содежржит информативные таблицы с описанием атрибутов перечисленных требований
    \item разделы 5, 6 - содержат преценденты и use-case системы
\end{itemize}
\section{Overall Description}
\subsection{Product functions}
Сайт должен содержать главную страницу с ссылками на другие разделы системы. Среди этих разделов должны быть:
\begin{enumerate}[itemsep=0.12ex]
    \item Клуб
    \item Как играть
    \item Для детей
    \item Новости
    \item Контакты
    \item Войти, Регистрация, забыли пароль?
    \item Топ 20
    \item Все пазлы
\end{enumerate}
После выбора пазла пользователь должен перенаправляться на интерактивный сайт, на котором изображение разбивается на выбранное им количество даталей. Пользователь должен иметь возможность выбирать деталь по одной и помещать ее на свободную доску, собирая пазл.
\subsection{User characteristics}
Целевой аудиторией продукта являются любители пазлов в возрасте 8-80 лет, которые имеют доступ к пк с установленным браузером, доступ в интернет и имеют базовые навыки владения пк.
Пользователи должны иметь возможность создавать аккаунты на сайте для сохранения своих достижений, виртуальной валюты и прочего.
Также система должна поддерживать специальный тип пользовательского аккаунта: администратор, который имеет возможность добавлять новые картинки в систему и модерировать комментарии.
\subsection{Assumptions and dependencies}
BigPuzzle.ru предполагает, что пользователи имеют доступ к современному веб-браузеру с интернет-соединением и могут создавать и управлять учетными записями с использованием электронной почты. Сайт также зависит от доступности и стабильности сторонних сервисов, таких как обработка платежей и доставка электронной почты.
\subsection{Constrains}
BigPuzzle.ru должен соответствовать всем соответствующим законам и правилам, связанным с онлайн-торговлей, защитой данных пользователей и конфиденциальностью. Сайт должен учитывать лицензии сторонних сервисов, используемых им, и подчиняться этим лицензиям. Сайт должен быть разработан с учетом возможности дальнейшего масштабирования.
\section{Specific Requirements}
\subsection{Functionality}
\subsubsection{Регистрация пользователя}
Система должна позволять пользователям регистрироваться на сайте, используя свои электронные почтовые адреса. При регистрации пользователю необходимо указывать свой email-адрес, после успешной регистрации он должен получить подтверждение на этот адрес.
\subsubsection{Поиск изображения для пазла}
Система должна позволять пользователям искать пазлы по различным критериям: их сложности, тегам. После поискового запроса подходящие изображения для пазлов должны отобразиться пользователю в формате, поддерживающем пагинацию
\subsubsection{Решение пазла}
После выбора изображения система должна позволить пользователю выбрать разбиение этого изображения на кусочки пазлов (8x8, 10x10, 12x12, и так далее). После выбора разбиения пользователь должен получать доступ к интерактивной игре, где можно выбрать кусочек и поместить его на свободное поля, формируя собранный пазл.
\subsubsection{Заработок виртуальных кредитов}
Система должна позволять пользователям зарабатывать виртуальные кредиты за прохождение пазлов и занесение своих результатов в таблицу лидеров.
\subsubsection{Система комментариев}
Система должна позволять пользователям оставлять комментарии к главной странице игры.
\subsubsection{Новостная лента}
Система должна иметь вкладку с новостной лентой, в которой отображаются последние новости, связанные с игрой.


\subsubsection{Покупка VIP-статуса}
Система должна позволять пользователям покупать VIP-статус для того, чтобы убрать рекламу во время пользования сайтом. VIP-статус может приобретаться по ситстеме подписки: на месяц, 6 месяцев и год.
\subsection{Usability}
\subsubsection{Обучаемость}
Система должна быть легко обучаемой для новых пользователей. Среднее время обучения пользователя интерфейсу не должно превышать 5 минут. Среднее время решения одного пазла зависит от количества кусочков в разбиении, но оно не должно превышать 10 минут при разбиении 18x18.
\subsubsection{Интуитивность интерфейса}
Интерфейс системы должен быть интуитивно понятным для пользователей. Кнопки и элементы управления должны быть ясными и доступными.
\subsubsection{Оформление сайта}
Оформление сайта должно быть приятным и привлекательным для пользователей. Дизайн должен быть современным и соответствовать тематике сайта.
\subsubsection{Понятность текстов}
Тексты на сайте должны быть понятными и доступными для всех пользователей. Язык должен быть легко воспринимаемым и не содержать сложных терминов и аббревиатур.
\subsubsection{Адаптивность сайта}
Сайт должен быть адаптивным и корректно отображаться на различных устройствах и разрешениях экранов.
\subsection{Reliability}
\subsubsection{Uptime}
Система должна быть доступна пользователям в течение 99.0\% времени в месяц, за исключением запланированных работ по обновлению и технического обслуживания.
\subsubsection{Backup}
Система должна поддерживать backup данных пользователей для восстановления после сбоев или атак.
\subsection{Performance}
\subsubsection{Количество одновременных пользователей}
Система должна поддерживать не менее 1000 одновременных пользователей без значительных уменьшений производительности
\subsubsection{Отклик}
Максимальное время ответа на запрос пользователя не должно превышать 2 секунд, а среднее время ответа должно быть около 500 миллисекунд.
\subsubsection{Обработка большого количества одновременных запросов}
Система должна иметь возможность обрабатывать до 1000 одновременных запросов в секунду без значительных уменьшений производительности.
\subsection{Design constraints}
\subsubsection{Используемые технологии}
Для создания клиентской стороны необходимо использовать JS, HTML, CSS.
\subsubsection{Использование сторонних сервисов}
Для оплаты vip-статуса наобходимо использовать сторонние сервисы. \\ 
Система интерактивного сбора пазла должна использовать Unity.
\subsection{Interfaces}
\subsubsection{User interfaces}
Пользовательский интерфейс главной страницы должен состоять из:
\begin{itemize}[itemsep=0.25ex]
    \item Шапки серого цвета, на которой расположены логотип, поля для регистрации и входа
    \item Кнопок, направляющих на секции системы, расположенных под шапкой
    \item Ссылок на социальные сети BigPuzzle.ru
    \item Изображений самых популярных пазлов
\end{itemize}
Пользовательский интерфейс сборки пазла должен состоять из:
\begin{itemize}[itemsep=0.25ex]
    \item Свободного белого поля для расположения пазлов
    \item Поля, в котором расположены ещё не поставленные пазлы
    \item Таймера, который засекает время сборки
\end{itemize}
\subsubsection{Hardware Interfaces}
Серверная часть должна состоять из высопроизводительной вычислительной системы.
Клиенская сторона состоит из мобильного устройства или пк с доступом в интернет и современным браузером.
\subsubsection{Software Interfaces}
Система должна иметь возможность использовать почтовые сервисы для регистрации и авторизации
\subsubsection{Communication Interfaces}
Для передачи информации в сети интернет система должна использовать протокол HTTP.
\subsection{Licensing Requirements}
Все права на публикацию и распространение информации с BigPuzzle.ru принадлежат авторам

\section{Атрибуты требований}
\subsection{Функциональные требования}
\begin{table}[H]
{\scriptsize
\begin{tabular}{|c|c|c|c|c|c|}
    \hline
    № & Требование & Приоритет & Трудоемкость (ч) & Стабильность & Риск \\
    \hline
    1 & Поиск по пазлам & 10 & 30 & Высокая & Высокий \\
    \hline
    2 & Решение пазла & 10 & 50 & Высокая & Высокий \\
    \hline
    3 & Регистрация пользьзователя & 8 & 4 & Высокая & Высокий \\
    \hline
    4 & Заработок виртуальных кредитов & 7 & 6 & Средняя & Низкий \\
    \hline
    5 & Покупка VIP-статуса & 7 & 15 & Средняя & Высокий \\
    \hline
    6 & Система комментариев & 5 & 8 & Низкая & Низкий \\
    \hline
    7 & Новостная лента & 4 & 5 & Низкая & Низкий \\
    \hline
    8 & Добавление нового изображения в систему & 10 & 10 & Высокая & Высокий \\
    \hline
\end{tabular}
}
\end{table}
\subsection{Нефункциональные требования}
\begin{table}[H]
{\scriptsize
\begin{tabular}{|c|c|c|c|c|c|}
    \hline
    № & Требование & Приоритет & Трудоемкость (ч) & Стабильность & Риск \\
    \hline
    1 & Обучаемость & 7 & 6 & Высокая & Низкий \\
    \hline
    2 & Интуитивность интерфейса & 10 & 30 & Высокая & Низкий \\
    \hline
    3 & Оформление сайта & 6 & 30 & Высокая & Низкий \\
    \hline
    4 & Понятность текстов & 9 & 6 & Высокая & Низкий \\
    \hline
    5 & Поддержание uptime & 10 & 40 & Средняя & Высокий \\
    \hline
    6 & Адаптивность сайта & 9 & 10 & Средняя & Низкий \\
    \hline
    7 & Количество одновременных пользователей & 10 & 40 & Высокая & Высокий \\
    \hline
    8 & Количество одновременных запросов & 10 & 40 & Высокая & Высокий \\
    \hline
    9 & Создание backup & 9 & 10 & Высокая & Высокий \\
    \hline
    
\end{tabular}
}
\end{table}
\section{Преценденты}

\begin{table}[H]
\begin{tabular}{|l|}
    \hline
    \textbf{Прецендент:} Просмотр главной страницы \\
    \hline
    \textbf{ID:} 1 \\
    \hline
    \textbf{Краткое описание:} пользователь открывает главную страницу для поиска нужного ему \\
    контента, кнопки или пазла. \\
    \hline
    \textbf{Главный актер:} пользователь \\
    \hline
    \textbf{Второстепенные актёры:} нет \\
    \hline
    \textbf{Предусловия:} нет \\
    \hline 
    \textbf{Основной поток:} Пользователь открывает главную страницу, скроллит её \\и нажимает на кнопки.\\
    \hline
\end{tabular}
\end{table}

\begin{table}[H]
\begin{tabular}{|l|}
    \hline
    \textbf{Прецендент: игра в пазл} \\
    \hline
    \textbf{ID: 2} \\
    \hline
    \textbf{Краткое описание:} после выбора картинки пользователь начинает игру в пазл \\
    \hline
    \textbf{Главный актер:} Пользователь \\
    \hline
    \textbf{Второстепенные актёры:} нет \\
    \hline
    \textbf{Предусловия:} нет \\
    \hline 
    \textbf{Основной поток:} \\
    1. Пользователь находит нужное ему изображение на главной странице \\
    или в секции поиска \\
    2. Пользователь нажимает на него и переходит на страницу с интерактивной игрой \\
    3. Пользователь выбирает нужное ему разбиение \\
    4. Пользователь использует панель сбоку, чтобы расположить пазлы на свободном поле \\
    5. После правильного расположения пазлов пользователю засчитывается победа \\
    \hline
\end{tabular}
\end{table}

\begin{table}[H]
\begin{tabular}{|l|}
    \hline
    \textbf{Прецендент:} регистрация\\
    \hline
    \textbf{ID:} 3 \\
    \hline
    \textbf{Краткое описание:} пользователь впервые заходит на сайт и регистрируется \\
    \hline
    \textbf{Главный актер:} пользователь \\
    \hline
    \textbf{Второстепенные актёры:} нет \\
    \hline
    \textbf{Предусловия:} пользователь ещё не зарегистрирован \\
    \hline 
    \textbf{Основной поток:} \\
    1. Пользователь нажимает на кнопку \enquote{Регистрация} и переходит в меню регистрации \\
    2. Пользователь вводит свою электронную почту и придуманный пароль \\
    3. Пользователь вводит код, пришедший ему на электронную почту, \\ его данные сохраняются в системе, и он успешно завершает регистрацию \\
    \hline
\end{tabular}
\end{table}

\begin{table}[H]
\begin{tabular}{|l|}
    \hline
    \textbf{Прецендент:} авторизация \\
    \hline
    \textbf{ID:} 4 \\
    \hline
    \textbf{Краткое описание:} пользователь заходит на сайт и регистрируется \\
    \hline
    \textbf{Главный актер:} пользователь  \\
    \hline
    \textbf{Второстепенные актёры:} нет \\
    \hline
    \textbf{Предусловия:} пользователь зарегистрирован на сайте \\
    \hline 
    \textbf{Основной поток:} \\
    1. Пользователь вводит адрес своей электронной почты и пароль в шапке главной \\
    страницы и нажимает на кнопку \enquote{войти} \\
    2. Система проверяет правильность введенных данных и позволяет пользователю \\
    авторизоваться, если данные верны \\
    3. Пользователь успешно авторизуется на сайте \\
    \hline
\end{tabular}
\end{table}


\begin{table}[H]
\begin{tabular}{|l|}
    \hline
    \textbf{Прецендент:} поиск картинки \\
    \hline
    \textbf{ID:} 5 \\
    \hline
    \textbf{Краткое описание:} пользователь ищет картинку по тегам \\
    \hline
    \textbf{Главный актер:} пользователь  \\
    \hline
    \textbf{Второстепенные актёры:} нет \\
    \hline
    \textbf{Предусловия:} нет \\
    \hline 
    \textbf{Основной поток:} \\
    1. Пользователь заходит на страницу поиска картинки по тегам и параметрам \\
    2. Пользователь отмечает несколько тегов \\
    3. Пользователю открывается лента картинок, отфильтрованных по выбранным тегам \\ 
    \hline
\end{tabular}
\end{table}

\begin{table}[H]
\begin{tabular}{|l|}
    \hline
    \textbf{Прецендент:} восстановление аккаунта \\
    \hline
    \textbf{ID:} 6 \\
    \hline
    \textbf{Краткое описание:} пользователь восстанавливает пароль от аккаунта \\
    \hline
    \textbf{Главный актер:} пользователь \\
    \hline
    \textbf{Второстепенные актёры:} нет \\
    \hline
    \textbf{Предусловия:} пользователь не помнит пароль от своего аккаунта \\
    \hline 
    \textbf{Основной поток:} \\
    1. Пользователь нажимает на кнопку \enquote{забыли пароль?} на главной странице \\ 
    2. Пользователь вводит адрес своей электронной почты на открывшейся странице \\
    3. Пользователь получает на электронную почту специальный код \\
    и вводит его на странице \\
    4. В открывшемся окне пользователь вводит новый пароль \\
    5. Новый пароль сохраняется в системе \\
    \hline
\end{tabular}
\end{table}

\begin{table}[H]
\begin{tabular}{|l|}
    \hline
    \textbf{Прецендент:} добавление нового изображения в систему \\
    \hline
    \textbf{ID:} 7 \\
    \hline
    \textbf{Краткое описание:} администратор добавляет новое изображение в систему \\
    \hline
    \textbf{Главный актер:} администратор \\
    \hline
    \textbf{Второстепенные актёры:} нет \\
    \hline
    \textbf{Предусловия:} нет \\
    \hline 
    \textbf{Основной поток:} \\
    1. Администратор входит в систему под специальным аккаунтом, расширяющим \\ возможности 
    авторизованных пользователей \\
    2. Администратор выбирает изображение и загружает его в систему \\
    3. Система сохраняет изображение и позволяет остальным пользователям находить это \\ 
    изображение в системе поиска \\
    \hline
\end{tabular}
\end{table}

\section{Use-case}


\begin{figure}[H]
  \centering
  \includegraphics[width=1\textwidth]{use-case.png}
\end{figure}

\section*{Вывод}
\addcontentsline{toc}{section}{Вывод}
Благодаря Software Requirements Specification можно очень точно предъявить требования к разрабатываемому ПО, а методология RUP и Use-Case диаграммы будут очень полезны при тестировании и выявлении каких-либо проблем на самых ранних стадииях разработки.

\end{document}

